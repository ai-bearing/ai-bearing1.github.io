\documentclass[10pt]{extarticle}
\title{}
\author{Avinash Iyer}
\date{}
\usepackage[shortlabels]{enumitem}

%font setup
%
%\usepackage[light,math]{anttor}
\usepackage[T1]{fontenc}
\usepackage[utf8]{inputenc}

%paper setup
\usepackage{geometry}
\geometry{letterpaper, portrait, margin=1in}
\usepackage{fancyhdr}

%symbols
\usepackage{amsmath}
\usepackage{mathtools}
\usepackage{amssymb}
\usepackage{hyperref}
\usepackage{gensymb}


%chemistry stuff
\usepackage[version=4]{mhchem}
\usepackage{chemfig}

%plotting
\usepackage{pgfplots}
\usepackage{tikz}

%\usepackage{natbib}

%graphics stuff
\usepackage{graphicx}
\graphicspath{ {./images/} }

%code stuff
%when using minted, make sure to add the -shell-escape flag
%you can use lstlisting if you don't want to use minted
%\usepackage{minted}
%\usemintedstyle{pastie}
%\newminted[javacode]{java}{frame=lines,framesep=2mm,linenos=true,fontsize=\footnotesize,tabsize=3,autogobble,}
%\newminted[cppcode]{cpp}{frame=lines,framesep=2mm,linenos=true,fontsize=\footnotesize,tabsize=3,autogobble,}

\usepackage{listings}
\usepackage{color}
\definecolor{dkgreen}{rgb}{0,0.6,0}
\definecolor{gray}{rgb}{0.5,0.5,0.5}
\definecolor{mauve}{rgb}{0.58,0,0.82}

\lstset{frame=tb,
	language=Java,
	aboveskip=3mm,
	belowskip=3mm,
	showstringspaces=false,
	columns=flexible,
	basicstyle={\small\ttfamily},
	numbers=none,
	numberstyle=\tiny\color{gray},
	keywordstyle=\color{blue},
	commentstyle=\color{dkgreen},
	stringstyle=\color{mauve},
	breaklines=true,
	breakatwhitespace=true,
	tabsize=3
}
% text + color boxes
\usepackage{tcolorbox}
\tcbuselibrary{breakable}
\newtcolorbox{problem}[1]{colback = white, title = {#1}, breakable}
\newtcolorbox{solution}{colback = white, colframe = black!75!white, title = Solution, breakable}
%including PDFs
\usepackage{pdfpages}
\setlength{\parindent}{0pt}

\pagestyle{fancy}
\fancyhf{}
\rhead{Avinash Iyer}
\lhead{Class Notes}
\begin{document}
  \section*{2.3}%
  \begin{problem}{Definition}
    A \textbf{weighted graph} $G$ is a graph alongside a function $w: E(G) \rightarrow \mathbb{R}^+$.\\

    If $G$ is a weighted graph and $H\subseteq G$, then, $w(H):=\sum\limits_{e\in E(H)} w(e)$.\\

    A \textbf{minimum weight spanning tree} (or MWST) is a spanning tree $T$ such that $w(T)$ is minimized among all possible spanning trees. In other words, $w(T) \leq w(T')~\forall T'\subseteq G$ where $T'$ is a spanning tree.
  \end{problem}
  \begin{problem}{Kruskal's Algorithm}
    \begin{description}[font=\normalfont\scshape]
      \item[Input] Weighted graph $G$ with $n$ vertices
      \item[Output] A MWST, $T^*$ if $G$ is connected, otherwise a message ``$G$ is not connected''
      \item[Step 1] Create a list of edges, $L_E$, in order from smallest weight to largest weight. Start $T^*$ with no edges but all vertices of $G$.
      \item[Step 2] If the number of edges in $T^*$ is strictly less than $n-1$ \textsc{and} if there are still edges in $L_E$, examine the first edge in $L_E$ (i.e., the edge with smallest weight), $e = \{a,b\}$
        \begin{description}[font=\normalfont\scshape]
          \item[Substep 2.1] If $a$ and $b$ are in different components of $T^*$, add $e$ to $T^*$ and remove $e$ from $L_E$. Return to \textsc{Step 2}.
          \item[Substep 2.2] If $a$ and $b$ are in the same component, remove $e$ from $L_E$ and do not add to $T^*$. Return to \textsc{Step 2}.
        \end{description}
      \item[Step 3] If the number of edges in $T^*$ is $n-1$, then output $T^*$, which is the MWST for $G$. Otherwise, the number of edges in $T^*$ is strictly less than $n-1$ and $G$ was not connected.
    \end{description}
  \end{problem}
  \begin{problem}{Example}
    We will find a MWST for the following graph:
    \begin{center}
      \includegraphics[width=5cm]{MWST_Example}
    \end{center}
    First, we create the following table of all the edges in $G$.
    \begin{center}
      \small
      \begin{tabular}{c|c}
        Edge & Cost \\
        \hline
        $af$ & $2$ \\
        $ab$ & $4$ \\
        $de$ & $5$ \\
        $ae$ & $6$ \\
        $ef$ & $7$ \\
        $bc$ & $8$ \\
        $cd$ & 8 \\
        $ad$ & 9\\
        $be$ & 9 \\
        $ac$ & 10 \\
        $bf$ & 13 \\
        $df$ & 15 \\
        $bd$ & 17 \\
        $cf$ & 19 \\
        $ce$ & 21
      \end{tabular}
    \end{center}
    We can read the following table describing the steps in Kruskal's algorithm from left to right (i.e., we check the edge of lowest weight, then we check the components, then we select our substep).
    \begin{center}
      \begin{tabular}{c|c|c}
        Edge & Components Of $T'$& Substep to be used\\
        \hline
        $af$ & $\{a\},\{b\},\{c\},\{d\},\{e\},\{f\}$ & 2.1 \\
        $ab$ & $\{a,f\},\{b\},\{c\},\{d\},\{e\}$ & 2.1 \\
        $de$ & $\{a,b,f\},\{c\},\{d\},\{e\}$ & 2.1 \\
        $ae$ & $\{a,b,f\},\{c\},\{d,e\}$ & 2.1 \\
        $ef$ & $\{a,b,d,e,f\},\{c\}$ & 2.2 \\
        $bc$ & $\{a,b,d,e,f\},\{c\}$ & 2.1 \\
        --- & $\{a,b,c,d,e,f\}$ & ---
      \end{tabular}
    \end{center}
  \end{problem}
  \begin{problem}{Proof of Kruskal's Algorithm}
    If $G$ is connected, then Kruskal's Algorithm produces a minimum weight spanning tree.
    \tcblower
    Let $G$ be a connected graph, and let $T_K$ be the output from Kruskal's algorithm on $G$. It is easy to check that $T_K$ is a spanning tree, since it is acyclic and has $n(G) - 1$ edges.\\

    Suppose $T^*$ is a MWST with largest edge intersection with $T_K$ --- in other words, $|E(T_K)\cap E(T^*)| \geq |E(T_K)\cap E(T')|$ for any other MWST $T'$.\\

    If $T^* = T_K$, then we are done, since $T_K$ is assumed to be a minimum weight spanning tree. Otherwise, assume toward contradiction that $T^*\neq T_K$. Let $e$ be the first edge chosen by Kruskal's algorithm that is not in $T^*$. Then, by a previous result, $\exists e'\in E(T^*) - E(T_K)$ such that $T' = T^* +e-e'$ is a spanning tree.\\

    We are assuming, however, that Kruskal's algorithm would choose $e$ over $e'$. Let $e_1,\dots,e_k$ be edges in $E(T_K)$ before $e$. Since $e_i$ was selected before $e$, we know that $e_i\in E(T^*)$ for each $i\in [k]$.\\

    Let $G_k = (V,\{e_1,\dots,e_k\})$. we are assuming that Kruskal's algorithm would not choose $e'$ for two reasons:
    \begin{itemize}
      \item If $e'$ shows up before $e$, then $e'$ would connect two vertices in $G_j = (V,\{e_1,\dots,e_j\})$. Since $G_j + e'\subseteq T^*$, then $T^*$ contains a cycle and isn't a spanning tree.
      \item If $e'$ shows up after $e$ in $L_E$, then $w(e') \geq w(e)$, meaning $w(T') \leq w(T^*)$, meaning that $w(T_K) \leq w(T^*)$, implying that $T_K$ is of a lower weight than $T^*$. 
    \end{itemize}
  \end{problem}
\end{document}
