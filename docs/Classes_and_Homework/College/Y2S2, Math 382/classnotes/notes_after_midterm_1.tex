\documentclass[10pt]{extarticle}
\title{}
\author{Avinash Iyer}
\date{}
\usepackage[shortlabels]{enumitem}

%font setup
%
%\usepackage[light,math]{anttor}
\usepackage{newpxtext,eulerpx}
\usepackage[T1]{fontenc}
\usepackage[utf8]{inputenc}

%paper setup
\usepackage{geometry}
\geometry{letterpaper, portrait, margin=1in}
\usepackage{fancyhdr}
\usepackage{soul}

%symbols
\usepackage{amsmath}
\usepackage{mathtools}
\usepackage{amssymb}
\usepackage{hyperref}
\usepackage{gensymb}


%chemistry stuff
\usepackage[version=4]{mhchem}
\usepackage{chemfig}

%plotting
\usepackage{pgfplots}
\usepackage{tikz}
\tikzset{middleweight/.style={pos = 0.5, fill=white}}
\tikzset{weight/.style={pos = 0.5, fill = white}}
\tikzset{lateweight/.style={pos = 0.75, fill = white}}
\tikzset{earlyweight/.style={pos = 0.25, fill=white}}

%\usepackage{natbib}

%graphics stuff
\usepackage{graphicx}
\graphicspath{ {./images/} }

%code stuff
%when using minted, make sure to add the -shell-escape flag
%you can use lstlisting if you don't want to use minted
%\usepackage{minted}
%\usemintedstyle{pastie}
%\newminted[javacode]{java}{frame=lines,framesep=2mm,linenos=true,fontsize=\footnotesize,tabsize=3,autogobble,}
%\newminted[cppcode]{cpp}{frame=lines,framesep=2mm,linenos=true,fontsize=\footnotesize,tabsize=3,autogobble,}

\usepackage{listings}
\usepackage{color}
\definecolor{dkgreen}{rgb}{0,0.6,0}
\definecolor{gray}{rgb}{0.5,0.5,0.5}
\definecolor{mauve}{rgb}{0.58,0,0.82}

\lstset{frame=tb,
	language=Java,
	aboveskip=3mm,
	belowskip=3mm,
	showstringspaces=false,
	columns=flexible,
	basicstyle={\small\ttfamily},
	numbers=none,
	numberstyle=\tiny\color{gray},
	keywordstyle=\color{blue},
	commentstyle=\color{dkgreen},
	stringstyle=\color{mauve},
	breaklines=true,
	breakatwhitespace=true,
	tabsize=3
}
% text + color boxes
\usepackage{tcolorbox}
\tcbuselibrary{breakable}
\newtcolorbox{problem}[1]{colback = white, title = {#1}, breakable}
\newtcolorbox{solution}{colback = white, colframe = black!75!white, title = Solution, breakable}
%including PDFs
\usepackage{pdfpages}
\setlength{\parindent}{0pt}

\pagestyle{fancy}
\fancyhf{}
\rhead{Avinash Iyer}
\lhead{Class Notes}
\begin{document}
  \section*{2.3}%
  \begin{problem}{Definition}
    A \textbf{weighted graph} $G$ is a graph alongside a function $w: E(G) \rightarrow \mathbb{R}^+$.\\

    If $G$ is a weighted graph and $H\subseteq G$, then, $w(H):=\sum\limits_{e\in E(H)} w(e)$.\\

    A \textbf{minimum weight spanning tree} (or MWST) is a spanning tree $T$ such that $w(T)$ is minimized among all possible spanning trees. In other words, $w(T) \leq w(T')~\forall T'\subseteq G$ where $T'$ is a spanning tree.
  \end{problem}
  \begin{problem}{Kruskal's Algorithm}
    \begin{description}[font=\normalfont\scshape]
      \item[Input] Weighted graph $G$ with $n$ vertices
      \item[Output] A MWST, $T^*$ if $G$ is connected, otherwise a message ``$G$ is not connected''
      \item[Step 1] Create a list of edges, $L_E$, in order from smallest weight to largest weight. Start $T^*$ with no edges but all vertices of $G$.
      \item[Step 2] If the number of edges in $T^*$ is strictly less than $n-1$ \textsc{and} if there are still edges in $L_E$, examine the first edge in $L_E$ (i.e., the edge with smallest weight), $e = \{a,b\}$
        \begin{description}[font=\normalfont\scshape]
          \item[Substep 2.1] If $a$ and $b$ are in different components of $T^*$, add $e$ to $T^*$ and remove $e$ from $L_E$. Return to \textsc{Step 2}.
          \item[Substep 2.2] If $a$ and $b$ are in the same component, remove $e$ from $L_E$ and do not add to $T^*$. Return to \textsc{Step 2}.
        \end{description}
      \item[Step 3] If the number of edges in $T^*$ is $n-1$, then output $T^*$, which is the MWST for $G$. Otherwise, the number of edges in $T^*$ is strictly less than $n-1$ and $G$ was not connected.
    \end{description}
  \end{problem}
  \begin{problem}{Example}
    We will find a MWST for the following graph:
    \begin{center}
      \includegraphics[width=5cm]{MWST_Example}
    \end{center}
    First, we create the following table of all the edges in $G$.
    \begin{center}
      \small
      \begin{tabular}{c|c}
        Edge & Cost \\
        \hline
        $af$ & $2$ \\
        $ab$ & $4$ \\
        $de$ & $5$ \\
        $ae$ & $6$ \\
        $ef$ & $7$ \\
        $bc$ & $8$ \\
        $cd$ & 8 \\
        $ad$ & 9\\
        $be$ & 9 \\
        $ac$ & 10 \\
        $bf$ & 13 \\
        $df$ & 15 \\
        $bd$ & 17 \\
        $cf$ & 19 \\
        $ce$ & 21
      \end{tabular}
    \end{center}
    We can read the following table describing the steps in Kruskal's algorithm from left to right (i.e., we check the edge of lowest weight, then we check the components, then we select our substep).
    \begin{center}
      \begin{tabular}{c|c|c}
        Edge & Components Of $T'$& Substep to be used\\
        \hline
        $af$ & $\{a\},\{b\},\{c\},\{d\},\{e\},\{f\}$ & 2.1 \\
        $ab$ & $\{a,f\},\{b\},\{c\},\{d\},\{e\}$ & 2.1 \\
        $de$ & $\{a,b,f\},\{c\},\{d\},\{e\}$ & 2.1 \\
        $ae$ & $\{a,b,f\},\{c\},\{d,e\}$ & 2.1 \\
        $ef$ & $\{a,b,d,e,f\},\{c\}$ & 2.2 \\
        $bc$ & $\{a,b,d,e,f\},\{c\}$ & 2.1 \\
        --- & $\{a,b,c,d,e,f\}$ & ---
      \end{tabular}
    \end{center}
  \end{problem}
  \begin{problem}{Proof of Kruskal's Algorithm}
    If $G$ is connected, then Kruskal's Algorithm produces a minimum weight spanning tree.
    \tcblower
    Let $G$ be a connected graph, and let $T_K$ be the output from Kruskal's algorithm on $G$. It is easy to check that $T_K$ is a spanning tree, since it is acyclic and has $n(G) - 1$ edges.\\

    Suppose $T^*$ is a MWST with largest edge intersection with $T_K$ --- in other words, $|E(T_K)\cap E(T^*)| \geq |E(T_K)\cap E(T')|$ for any other MWST $T'$.\\

    If $T^* = T_K$, then we are done, since $T_K$ is assumed to be a minimum weight spanning tree. Otherwise, assume toward contradiction that $T^*\neq T_K$. Let $e$ be the first edge chosen by Kruskal's algorithm that is not in $T^*$. Then, by a previous result, $\exists e'\in E(T^*) - E(T_K)$ such that $T' = T^* +e-e'$ is a spanning tree.\\

    We are assuming, however, that Kruskal's algorithm would choose $e$ over $e'$. Let $e_1,\dots,e_k$ be edges in $E(T_K)$ before $e$. Since $e_i$ was selected before $e$, we know that $e_i\in E(T^*)$ for each $i\in [k]$.\\

    Let $G_k = (V,\{e_1,\dots,e_k\})$. we are assuming that Kruskal's algorithm would not choose $e'$ for two reasons:
    \begin{itemize}
      \item If $e'$ shows up before $e$, then $e'$ would connect two vertices in $G_j = (V,\{e_1,\dots,e_j\})$. Since $G_j + e'\subseteq T^*$, then $T^*$ contains a cycle and isn't a spanning tree.
      \item If $e'$ shows up after $e$ in $L_E$, then $w(e') \geq w(e)$, meaning $w(T') \leq w(T^*)$, meaning that $w(T_K) \leq w(T^*)$, implying that $T_K$ is of a lower weight than $T^*$. 
    \end{itemize}
  \end{problem}
  \begin{problem}{Dijkstra's Algorithm}
    We want to find an algorithm to find the shortest path between two vertices in a weighted graph --- Dijkstra's Algorithm can solve this.\\

    We know that if $P$ is the shortest $u,v$ path, and $x\in P$, then following $P$ from $u$ to $x$ will yield the shortest $u,x$ path.
    \begin{description}[font=\normalfont\scshape]
      \item[Input] A weighted graph, $G$, and $u\in V(G)$
      \item[Output] For each $z\in V(G)$, the distance $d(u,z)$.
      \item[Initialization]Extend the weight function such that if $xy\notin E(G)$, then $w(xy) = \infty$. Create $S$ that contains all vertices whose distances from $u$ are known. Let $S := \{u\}$. Let $t:V\rightarrow \mathbb{R}^+ \cup \{0,\infty\}$ which will keep track of the tentative distance between $u$ and $z$. Let $t(z) := w(uz)$ for all $z\neq u$, and $t(u) := 0$.
      \item[Condition to Terminate Loop] If $t(z) = \infty$ for all $z\notin S$ \textsc{or} $S = V$, then go to end.
      \item[Loop] Else, pick $v\in V-S$ such that $t(v) = \min_{z\notin S} t(z)$. Add $v$ to $S$. Explore the edges from $v$ to update tentative distances; for each edge $vz$ with $z\notin S$, $t(z) := \min\{t(z),t(v) + w(vz)\}$.
      \item[End] Set $d(u,v) = t(v)~\forall v\in V$.
    \end{description}
    On the following weighted graph, we can do Dijkstra's algorithm as follows:
    \begin{center}
      \begin{tikzpicture}
        \filldraw (-2,0) circle (2pt)
                  (-1,1) circle (2pt)
                  (-1,-1) circle (2pt)
                  (2,0) circle (2pt)
                  (1,-1) circle (2pt)
                  (1,1) circle (2pt);
        \node[anchor = east] (u) at (-2,0) {$u$};
        \node[anchor = south east] (a) at (-1,1) {$a$};
        \node[anchor = north east] (b) at (-1,-1) {$b$};
        \node[anchor = west] (e) at (2,0) {$e$};
        \node[anchor = north west] (c) at (1,-1) {$c$};
        \node[anchor = south west] (d) at (1,1) {$d$};
        \draw (-2,0) -- node[middleweight]{\small 1} (-1,1) -- node[middleweight]{\small 1}(1,1) -- node[middleweight]{\small 2}(2,0);
        \draw (-2,0) --node[middleweight]{\small 3}(-1,-1) --node[middleweight]{\small 5} (1,-1) -- node[middleweight]{\small 6} (2,0);
        \draw (-1,-1) --node[earlyweight]{\small 4}(1,1);
        \draw (-1,1) --node[earlyweight]{\small 4}(1,-1);
      \end{tikzpicture}
    \end{center}
  \end{problem}
  \begin{problem}{Dijkstra's Algorithm: Worked Example}
    We let $S = \{u\}$, and include our tentative distances for the first step.
    \begin{center}
      \begin{tikzpicture}
        \filldraw (-2,0) circle (2pt)
                  (-1,1) circle (2pt)
                  (-1,-1) circle (2pt)
                  (2,0) circle (2pt)
                  (1,-1) circle (2pt)
                  (1,1) circle (2pt);
        \node[anchor = east] (u) at (-2,0) {\small $t_u=0$};
        \node[anchor = south east] (a) at (-1,1) {\small $t_a = 1$};
        \node[anchor = north east] (b) at (-1,-1) {\small $t_b = 3$};
        \node[anchor = west] (e) at (2,0) {\small $t_e = \infty$};
        \node[anchor = north west] (c) at (1,-1) {\small $t_d = \infty$};
        \node[anchor = south west] (d) at (1,1) {\small $t_c = \infty$};
        \draw (-2,0) -- node[middleweight]{\small 1} (-1,1) -- node[middleweight]{\small 1}(1,1) -- node[middleweight]{\small 2}(2,0);
        \draw (-2,0) --node[middleweight]{\small 3}(-1,-1) --node[middleweight]{\small 5} (1,-1) -- node[middleweight]{\small 6} (2,0);
        \draw (-1,-1) --node[earlyweight]{\small 4}(1,1);
        \draw (-1,1) --node[earlyweight]{\small 4}(1,-1);
      \end{tikzpicture}
    \end{center}
    Now, $S = \{u,a\}$ because $t_a \leq t_b$. We now start from $a$ and include our tentative distances.
    \begin{center}
      \begin{tikzpicture}
        \filldraw (-2,0) circle (2pt)
                  (-1,1) circle (2pt)
                  (-1,-1) circle (2pt)
                  (2,0) circle (2pt)
                  (1,-1) circle (2pt)
                  (1,1) circle (2pt);
        \node[anchor = east] (u) at (-2,0) {\small $t_u=0$};
        \node[anchor = south east] (a) at (-1,1) {\small $t_a = 1$};
        \node[anchor = north east] (b) at (-1,-1) {\small $t_b = 3$};
        \node[anchor = west] (e) at (2,0) {\small $t_e = \infty$};
        \node[anchor = north west] (c) at (1,-1) {\small $t_c = 5$};
        \node[anchor = south west] (d) at (1,1) {\small $t_d = 6$};
        \draw (-2,0) -- node[middleweight]{\small 1} (-1,1) -- node[middleweight]{\small 1}(1,1) -- node[middleweight]{\small 2}(2,0);
        \draw (-2,0) --node[middleweight]{\small 3}(-1,-1) --node[middleweight]{\small 5} (1,-1) -- node[middleweight]{\small 6} (2,0);
        \draw (-1,-1) --node[earlyweight]{\small 4}(1,1);
        \draw (-1,1) --node[earlyweight]{\small 4}(1,-1);
      \end{tikzpicture}
    \end{center}
    Next, we include $b$ into $S$, making it $\{u,a,b\}$. We then check our tentative distances from $b$, where we find that they are no better than tentative distances from $a$, so we keep the tentative distances on $c$ and $d$ as what they were with $a$.
    \begin{center}
      \begin{tikzpicture}
        \filldraw (-2,0) circle (2pt)
                  (-1,1) circle (2pt)
                  (-1,-1) circle (2pt)
                  (2,0) circle (2pt)
                  (1,-1) circle (2pt)
                  (1,1) circle (2pt);
        \node[anchor = east] (u) at (-2,0) {\small $t_u=0$};
        \node[anchor = south east] (a) at (-1,1) {\small $t_a = 1$};
        \node[anchor = north east] (b) at (-1,-1) {\small $t_b = 3$};
        \node[anchor = west] (e) at (2,0) {\small $t_e = \infty$};
        \node[anchor = north west] (c) at (1,-1) {\small $t_c =5$};
        \node[anchor = south west] (d) at (1,1) {\small $t_d = 6$};
        \draw (-2,0) -- node[middleweight]{\small 1} (-1,1) -- node[middleweight]{\small 1}(1,1) -- node[middleweight]{\small 2}(2,0);
        \draw (-2,0) --node[middleweight]{\small 3}(-1,-1) --node[middleweight]{\small 5} (1,-1) -- node[middleweight]{\small 6} (2,0);
        \draw (-1,-1) --node[earlyweight]{\small 4}(1,1);
        \draw (-1,1) --node[earlyweight]{\small 4}(1,-1);
      \end{tikzpicture}
    \end{center}
    Next, we include $c$ into $S$, making it $\{u,a,b,c\}$, and we update our tentative distances.
    \begin{center}
      \begin{tikzpicture}
        \filldraw (-2,0) circle (2pt)
                  (-1,1) circle (2pt)
                  (-1,-1) circle (2pt)
                  (2,0) circle (2pt)
                  (1,-1) circle (2pt)
                  (1,1) circle (2pt);
        \node[anchor = east] (u) at (-2,0) {\small $t_u=0$};
        \node[anchor = south east] (a) at (-1,1) {\small $t_a = 1$};
        \node[anchor = north east] (b) at (-1,-1) {\small $t_b = 3$};
        \node[anchor = west] (e) at (2,0) {\small $t_e = 11$};
        \node[anchor = north west] (c) at (1,-1) {\small $t_c =5$};
        \node[anchor = south west] (d) at (1,1) {\small $t_d = 6$};
        \draw (-2,0) -- node[middleweight]{\small 1} (-1,1) -- node[middleweight]{\small 1}(1,1) -- node[middleweight]{\small 2}(2,0);
        \draw (-2,0) --node[middleweight]{\small 3}(-1,-1) --node[middleweight]{\small 5} (1,-1) -- node[middleweight]{\small 6} (2,0);
        \draw (-1,-1) --node[earlyweight]{\small 4}(1,1);
        \draw (-1,1) --node[earlyweight]{\small 4}(1,-1);
      \end{tikzpicture}
    \end{center}
    Finally, we include $d$ and update tentative distances:
    \begin{center}
      \begin{tikzpicture}
        \filldraw (-2,0) circle (2pt)
                  (-1,1) circle (2pt)
                  (-1,-1) circle (2pt)
                  (2,0) circle (2pt)
                  (1,-1) circle (2pt)
                  (1,1) circle (2pt);
        \node[anchor = east] (u) at (-2,0) {\small $t_u=0$};
        \node[anchor = south east] (a) at (-1,1) {\small $t_a = 1$};
        \node[anchor = north east] (b) at (-1,-1) {\small $t_b = 3$};
        \node[anchor = west] (e) at (2,0) {\small $t_e = 8$};
        \node[anchor = north west] (c) at (1,-1) {\small $t_c =5$};
        \node[anchor = south west] (d) at (1,1) {\small $t_d = 6$};
        \draw (-2,0) -- node[middleweight]{\small 1} (-1,1) -- node[middleweight]{\small 1}(1,1) -- node[middleweight]{\small 2}(2,0);
        \draw (-2,0) --node[middleweight]{\small 3}(-1,-1) --node[middleweight]{\small 5} (1,-1) -- node[middleweight]{\small 6} (2,0);
        \draw (-1,-1) --node[earlyweight]{\small 4}(1,1);
        \draw (-1,1) --node[earlyweight]{\small 4}(1,-1);
      \end{tikzpicture}
    \end{center}
    In order to find the direct paths, we can add arrows along the edges that were selected by Dijkstra's algorithm.\\

    The proof can be outlined as follows:
    \begin{description}[font=\normalfont\scshape]
      \item[If $z\in S$:] then, $t(z) = d(u,z)$.
      \item[Else if $z\notin S$:] then, $t(z)$ is the length of a shortest $u,z$ path $P$ such that $V(P-z)\subseteq S$.
    \end{description}
  \end{problem}
  \section*{3.1}%
  \begin{problem}{Matchings}
    In a simple graph, a \textbf{matching} $M$ is a set of pairwise disjoint edges. In other words, $\forall e_i,e_j\in M$, then $e_i\cap e_j = \emptyset$. In an arbitrary graph, a matching is a set of non-loop edges with no shared endpoints. For example, the thick edges are a matching.
    \begin{center}
      \begin{tikzpicture}
        \filldraw (1,1) circle (2pt)
              (1,-1) circle (2pt)
              (-1,1) circle (2pt)
              (-1,-1) circle (2pt);
        \draw (1,1) -- (1,-1) -- (-1,-1) -- (-1,1)-- cycle;
        \draw[very thick] (1,1) -- (-1,-1);
        \draw[very thick] (-1,1) -- (1,-1);
      \end{tikzpicture}
    \end{center}
    If $v$ is incident to an edge, then $v$ is \textbf{saturated} by $M$, otherwise it is unsatured by $M$.\\

    A \textbf{perfect matching} is a matching that saturates every vertex. We know that all graphs with perfect matchings have even number of vertices, but the alternative case may not be true (for example, we might consider a graph with an isolated vertex).\\

    A \textbf{maximal matching} is a matching that cannot be enlarged by adding any other edges. A \textbf{maximum} matching is a matching of maximum size among all matchings. In other words, if $M^*$ is a maximum matching, then $|M^*| \geq |M|~\forall M\in G$.
    \begin{itemize}
      \item Not every maximal matching is a maximum matching.
      \item However, every maximum matching is a maximal matching (because you cannot extend a maximum matching by definition).
    \end{itemize}
  \end{problem}
  \begin{problem}{Alternating and Augmenting Paths}
    Given a matching $M$, an $M$-\textbf{alternating path} is a path that alternates between edges in $M$ and edges not in $M$. An $M$-alternating path whose endpoints are unsaturated by $M$ is an $M$-\textbf{augmenting} path.
    \begin{center}
      An $M$-alternating path: \\
      \vspace{10pt}
      \begin{tikzpicture}
        \filldraw (-2,0) circle (2pt)
                  (-1,1) circle (2pt)
                  (0,0) circle (2pt)
                  (1,1) circle (2pt)
                  (2,0) circle (2pt);
        \draw (-2,0) -- (-1,1);
        \draw[very thick] (-1,1) -- (0,0);
        \draw (0,0) -- (1,1);
        \draw[very thick] (1,1) -- (2,0);
      \end{tikzpicture}\\
      \vspace{20pt}
      An $M$-augmenting path:\\
      \vspace{10pt}
      \begin{tikzpicture}
        \filldraw (-2,0) circle (2pt)
                  (-1,1) circle (2pt)
                  (0,0) circle (2pt)
                  (1,1) circle (2pt)
                  (2,0) circle (2pt)
                  (3,1) circle (2pt);
        \draw (-2,0) -- (-1,1);
        \draw[very thick] (-1,1) -- (0,0);
        \draw (0,0) -- (1,1);
        \draw[very thick] (1,1) -- (2,0);
        \draw (2,0) -- (3,1);
      \end{tikzpicture}
    \end{center}
    If $M$ is a matching and there exists an $M$-augmenting path in $G$, then $M$ is not a maximum matching (as in the path $P$ that contains $M$, you can switch the matching edges as follows)
    \begin{center}
      \begin{tikzpicture}
        \filldraw (-2,0) circle (2pt)
                  (-1,1) circle (2pt)
                  (0,0) circle (2pt)
                  (1,1) circle (2pt)
                  (2,0) circle (2pt)
                  (3,1) circle (2pt);
        \draw[very thick] (-2,0) -- (-1,1);
        \draw (-1,1) -- (0,0);
        \draw[very thick](0,0) -- (1,1);
        \draw (1,1) -- (2,0);
        \draw[very thick] (2,0) -- (3,1);
      \end{tikzpicture}
    \end{center}
  \end{problem}
  \begin{problem}{Symmetric Difference}
    The \textbf{symmetric difference} $G\vartriangle H$ is the subgraph of $G\cup H$ whose edges are the edges of $G\cup H$ that appear in exactly one of $G$ and $H$. A picture of an example of a symmetric difference between matchings is shown below.
    \begin{center}
      \includegraphics[width=12cm]{symmetric_difference}
    \end{center}
  \end{problem}
  \begin{problem}{Theorems and Lemmas}
    \begin{problem}{Lemma 3.1.9}
      Every component of the symmetric difference of two matchings is a path or an even cycle.
      \tcblower
      Let $M$ and $M'$ be two matchings. Each vertex is incident to at most $1$ edge in $M$ and at most one edge in $M'$. Thus, $d(v)\leq 2$ for each vertex in $M$ and $M'$. Therefore, each component is either a path or a cycle.\\

      If a component is a cycle, then it must be even because the edges of the cycle must alternate between $M$ and $M'$.
    \end{problem}
    \begin{problem}{Theorem 3.1.10}
      A matching $M$ in a graph $G$ is a \textit{maximum} matching if and only if $G$ has no $M$-augmenting path.
      \tcblower
      \begin{description}[font=\normalfont\scshape]
        \item[($\Rightarrow$)] Suppose $G$ has an $M$-augmenting path, $P$. Exchange the edges of $P$ in $M$ with the edges of $P$ not in $M$. This action increases the size of the matching by $1$, meaning that $M$ was not a maximum matching initially.
        \item[$\Leftarrow$] Suppose $M$ is not a maximum matching. Let $M'$ be a matching with more edges than $M$ (i.e., $|M'| > |M|$). Let $H = (V(G),M\vartriangle M')$. By Lemma 3.1.9, we know that each component of $H$ is either a path or an even cycle. Since $|M'| > |M|$, there is a component of $H$ that contains more edges from $M'$ than edges from $M$, which means it cannot be an even cycle --- therefore, this component must be a path that alternates between $M'$ and $M$. Because there are more edges from $M'$ than $M$ in this path, meaning this path is an $M$-augmenting path.
      \end{description}
    \end{problem}
  \end{problem}
  \begin{problem}{Perfect Matchings}
    Let $G = (X,E,Y)$ be a bipartite graph. A matching that saturates $X$ is an $X$-\textbf{perfect matching}.\\

    If there exists a set of vertices $S\subseteq X$ such that $|S| > |N(S)|$, then it is impossible for $G$ to have an $X$-perfect matching.
    \begin{problem}{Theorem 3.1.11}
      A bipartite graph $G = (X,E,Y)$ has an $X$-perfect matching if and only if $|S| \leq |N(S)|$ for all $S\subseteq X$.
      \tcblower
      \begin{description}[font=\normalfont\scshape]
        \item[($\Rightarrow$)] If $G$ has an $X$-perfect matching, then each vertex in $X$ is matched to a distinct vertex in $N(S)$. Thus, $|S| \leq |N(S)|$.
        \item[($\Leftarrow$)] 
      \end{description}
    \end{problem}
  \end{problem}
\end{document}
