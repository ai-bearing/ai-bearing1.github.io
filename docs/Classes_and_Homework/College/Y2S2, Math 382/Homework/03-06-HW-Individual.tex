\documentclass[8pt]{extarticle}
\title{}
\author{Avinash Iyer}
\date{}
\usepackage[shortlabels]{enumitem}

%font setup
%
%\usepackage[math]{anttor}

%paper setup
\usepackage{geometry}
\geometry{letterpaper, portrait, margin=1in}
\usepackage{fancyhdr}

%symbols
\usepackage{amsmath}
\usepackage{mathtools}
\usepackage{amssymb}
\usepackage{hyperref}
\usepackage{gensymb}

\usepackage[T1]{fontenc}
\usepackage[utf8]{inputenc}

%chemistry stuff
\usepackage[version=4]{mhchem}
\usepackage{chemfig}

%plotting
\usepackage{pgfplots}
\usepackage{tikz}

%\usepackage{natbib}

%graphics stuff
\usepackage{graphicx}
\graphicspath{ {./images/} }

%code stuff
%when using minted, make sure to add the -shell-escape flag
%you can use lstlisting if you don't want to use minted
%\usepackage{minted}
%\usemintedstyle{pastie}
%\newminted[javacode]{java}{frame=lines,framesep=2mm,linenos=true,fontsize=\footnotesize,tabsize=3,autogobble,}
%\newminted[cppcode]{cpp}{frame=lines,framesep=2mm,linenos=true,fontsize=\footnotesize,tabsize=3,autogobble,}

\usepackage{listings}
\usepackage{color}
\definecolor{dkgreen}{rgb}{0,0.6,0}
\definecolor{gray}{rgb}{0.5,0.5,0.5}
\definecolor{mauve}{rgb}{0.58,0,0.82}

\lstset{frame=tb,
	language=Java,
	aboveskip=3mm,
	belowskip=3mm,
	showstringspaces=false,
	columns=flexible,
	basicstyle={\small\ttfamily},
	numbers=none,
	numberstyle=\tiny\color{gray},
	keywordstyle=\color{blue},
	commentstyle=\color{dkgreen},
	stringstyle=\color{mauve},
	breaklines=true,
	breakatwhitespace=true,
	tabsize=3
}
% text + color boxes
\usepackage{tcolorbox}
\tcbuselibrary{breakable}
\newtcolorbox{problem}[1]{colback = white, title = {#1}, breakable}
\newtcolorbox{solution}{colback = white, colframe = black!75!white, title = Solution, breakable}
%including PDFs
\usepackage{pdfpages}
\setlength{\parindent}{0pt}

\pagestyle{fancy}
\fancyhf{}
\rhead{Avinash Iyer}
\lhead{Homework Section 2.1, Individual}
\begin{document}{
  \begin{problem}{2.1.2}
    Let $G$ be a graph:
    \begin{enumerate}[(a)]
      \item Prove that $G$ is a tree if and only if $G$ is connected and every edge is a cut-edge.
      \item Prove that $G$ is a tree if and only if adding any edge with endpoints in $V(G)$ creates exactly one cycle.
    \end{enumerate}
  \end{problem}
  \begin{solution}
    \begin{tcolorbox}[colback = white, title = (a), breakable]
      \begin{description}[font=\normalfont\scshape]
        \item[($\Rightarrow$)] Let $G$ be a tree. Thus, $G$ is connected (by definition), and acyclic. Since $G$ is acyclic, this means that there are no edges within cycles, so by definition, every edge is a cut-edge.
        \item[($\Leftarrow$)] Let $G$ be a connected graph such that every edge is a cut-edge. Since there are no non-cut-edge edges, this means there are no cycles in $G$, so $G$ is a connected acyclic graph, or a tree.
      \end{description}
    \end{tcolorbox}
    \begin{tcolorbox}[colback = white, title = (b), breakable]
      \begin{description}[font=\normalfont\scshape]
        \item[($\Rightarrow$)] Let $G$ be a tree, and let $e$ be an edge such that $e\notin E(G)$, and $e = uv$. Then, we create a cycle from the path $uTv + e$ --- since there is only one path $uTv$, this means that $uTv+e$ is a unique cycle.
        \item[($\Leftarrow$)] Suppose toward contradiction that adding $e$ to the tree $G$ yielded more than one cycle in the graph $G+e$. Then, the graph $G = G + e - e$ would have at least one cycle, as we deleted an edge from one cycle in a graph with more than one cycle. However, since we assumed that $G$ was a tree, we have reached a contradiction, meaning that $e$ added exactly one cycle to the tree $G$.
      \end{description}
    \end{tcolorbox}
  \end{solution}
  \begin{problem}{2.1.6}
    Let $T$ be a tree with average degree $a$. In terms of $a$, find $n(T)$.
  \end{problem}
  \begin{solution}
    \begin{align*}
      d_{\textrm{avg}} &= \frac{2e(T)}{n(T)}\\
                    a  &= \frac{2(n(T) - 1)}{n(T)} \\
                    an &= 2n-2 \\
              (a-2)n &= -2 \\
              n &= \boxed{\frac{-2}{a-2}} \tag*{Needs to be corroborated}
    \end{align*}
  \end{solution}
  \begin{problem}{2.1.7}
    Prove that every $n$-vertex graph with $m$ edges has at least $n-m+1$ cycles.
  \end{problem}
  \begin{problem}{2.1.12}
    Compute the diameter and radius of $K_{m,n}$.
  \end{problem}
  \begin{problem}{2.1.13}
    Prove that every graph with diameter $d$ has an independent set with at least $\lfloor\frac{1+d}{2}\rfloor$ vertices.
  \end{problem}
}\end{document}
