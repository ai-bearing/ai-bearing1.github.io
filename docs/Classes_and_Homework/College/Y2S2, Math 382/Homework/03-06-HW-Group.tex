\documentclass[10pt]{extarticle}
\title{}
\author{Avinash Iyer}
\date{}
\usepackage[shortlabels]{enumitem}

%font setup
%
%\usepackage[math]{anttor}

%paper setup
\usepackage{geometry}
\geometry{letterpaper, portrait, margin=1in}
\usepackage{fancyhdr}

%symbols
\usepackage{amsmath}
\usepackage{mathtools}
\usepackage{amssymb}
\usepackage{hyperref}
\usepackage{gensymb}

\usepackage[T1]{fontenc}
\usepackage[utf8]{inputenc}

%chemistry stuff
\usepackage[version=4]{mhchem}
\usepackage{chemfig}

%plotting
\usepackage{pgfplots}
\usepackage{tikz}

%\usepackage{natbib}

%graphics stuff
\usepackage{graphicx}
\graphicspath{ {./images/} }

%code stuff
%when using minted, make sure to add the -shell-escape flag
%you can use lstlisting if you don't want to use minted
%\usepackage{minted}
%\usemintedstyle{pastie}
%\newminted[javacode]{java}{frame=lines,framesep=2mm,linenos=true,fontsize=\footnotesize,tabsize=3,autogobble,}
%\newminted[cppcode]{cpp}{frame=lines,framesep=2mm,linenos=true,fontsize=\footnotesize,tabsize=3,autogobble,}

\usepackage{listings}
\usepackage{color}
\definecolor{dkgreen}{rgb}{0,0.6,0}
\definecolor{gray}{rgb}{0.5,0.5,0.5}
\definecolor{mauve}{rgb}{0.58,0,0.82}

\lstset{frame=tb,
	language=Java,
	aboveskip=3mm,
	belowskip=3mm,
	showstringspaces=false,
	columns=flexible,
	basicstyle={\small\ttfamily},
	numbers=none,
	numberstyle=\tiny\color{gray},
	keywordstyle=\color{blue},
	commentstyle=\color{dkgreen},
	stringstyle=\color{mauve},
	breaklines=true,
	breakatwhitespace=true,
	tabsize=3
}
% text + color boxes
\usepackage{tcolorbox}
\tcbuselibrary{breakable}
\newtcolorbox{problem}[1]{colback = white, title = {#1}, breakable}
\newtcolorbox{solution}{colback = white, colframe = black!75!white, title = Solution, breakable}
%including PDFs
\usepackage{pdfpages}
\setlength{\parindent}{0pt}

\pagestyle{fancy}
\fancyhf{}
\rhead{Ling Chen, Avinash Iyer, Nora Manukyan, Vincent Ng}
\lhead{Homework Section 2.1, Group}
\begin{document}{
\begin{problem}{2.1.22}
  Let $T$ be an $n$-vertex tree with one vertex of each degree $2\leq i \leq k$; the remaining $n-k+1$ vertices are leaves. Determine $n$ in terms of $k$.
  \tcblower
  We will find the number of vertices in $T$ by finding the number of edges in $T$ and adding $1$. For $2\leq i\leq k$ corresponding to each of the non-leaf vertices, summation yields $\frac{k(k+1)}{2}-1$ edges. However, this scheme double-counts each edge, so we have to subtract the $k-2$ edges connecting the $k-1$ non-leaf vertices, yielding $\frac{k(k+1)}{2}-k+1$ edges. Finally, because $T$ is a tree, we get that $T$ has $\frac{k(k+1)}{2}-k+2$ vertices.
\end{problem}
\begin{problem}{2.1.27}
  Let $d_1,\dots,d_n$ be positive integers with $n\geq 2$. Prove that there exists a tree with vertex degrees $d_1,\dots,d_n$ if and only if $\sum d_i = 2n-2$.
  \tcblower
  \begin{description}[font=\normalfont\scshape]
    \item[($\Rightarrow$)] Suppose that for some tree $T$, $d_1,\dots,d_n$ are the degrees of the vertices of the tree. Since $T$ is a tree, this means $e(G) = n-1$, and $\sum d_i = 2e(G)$, meaning $\sum d_i = 2(n-1) = 2n-2$.
    \item[($\Leftarrow$)] Suppose that $\sum d_i = 2n-2$ for $d_1,\dots,d_n$ corresponding to the degrees of the vertices in $G$. By a previous result, we know that $\sum d_i = 2e(G)$, meaning that $\sum d_i = 2(n-1)$, implying that $e(G) = n-1$. We can find a tree $G$ with $n-1$ edges by letting $G$ be connected with $n-1$ edges.
  \end{description}
\end{problem}
\begin{problem}{2.1.33}
  Let $G$ be a connected $n$-vertex graph. Prove that $G$ has exactly one cycle if and only if $G$ has exactly $n$ edges.
  \tcblower
  \begin{description}[font=\normalfont\scshape]
    \item[($\Rightarrow$)] Let $G$ be a connected $n$-vertex graph with exactly one cycle. If we delete an edge from this cycle, then $G-e$ is acyclic, as well as connected (since $e$ is not a cut-edge), so $G-e$ has $n-1$ edges. Adding back $e$, we get that $G$ has $n$ edges.
    \item[($\Leftarrow$)] Let $G$ be a connected $n$-vertex graph with $n$ edges. Then, $G$ contains a spanning tree that contains all $n$ vertices. Therefore, $T\subseteq G$ contains $n-1$ edges. By adding another edge, we get that $e(G) = e(T) + 1 = n-1+1$. Thus, $G$ has exactly one cycle.
  \end{description}
\end{problem}
\begin{problem}{2.1.34}
  Let $T$ be a tree with $k$ edges, and let $G$ be a $n$-vertex simple graph with more than $n(k-1)-{ k \choose 2 }$ edges. Use Proposition $2.1.8$ to prove that $T\subseteq G$ if $n>k$.
  \tcblower
  We will use induction to prove that $T\subseteq G$ as follows:
  \begin{description}[font = \normalfont\scshape]
    \item[Base Case] Suppose $n = k+1$. Then, we can find the following:
      \begin{align*}
        e(G) &> (k+1)(k-1) - {k\choose 2}\\
        e(G) &> (k^2-1) - \frac{k(k-1)}{2}\\
        e(G) &> \frac{k^2 - 1}{2} + \frac{k^2-1 - (k^2-k)}{2} \\
        e(G) &> \frac{k^2 + k}{2} - 1\\
        e(G) &> \frac{k(k+1)}{2} - 1
      \end{align*}
      This means $e(G) = \frac{k(k+1)}{2}$ in the base case, meaning $G$ is the complete graph on $k+1$ vertices, where $\delta(G) = k$. By Theorem 2.1.8, we know that $T\subseteq G$.
    \item[Inductive Hypothesis] If $n > k+1$, $e(G) > n(k-1) - {k\choose 2}$, then either $\delta(G) \geq k$ or, if $\delta(G) < k$, then $e(G-x) > (n-1)(k-1) - {k\choose 2}$ for $\delta(G) = d(x)$.
    \item[Proof] If $\delta(G) \geq k$, then we know by Theorem 2.1.8 that $T\subseteq G$. Otherwise, suppose $\delta(G) < k$, and let $d(x) = \delta(G)$. Let $G' = G-x$.
      \begin{align*}
        e(G') &= e(G) - \delta(G)\\
        e(G') &\geq e(G) - (k-1) \\
        e(G') &> n(k-1) - {k\choose 2} - (k-1) \\
        e(G') &> (n-1)(k-1) - {k\choose 2}
      \end{align*}
      Therefore, the inductive hypothesis is proven.
  \end{description}
\end{problem}
}\end{document}
