\documentclass[9pt]{extarticle}
\title{}
\author{Avinash Iyer}
\date{}

%font setup
%
%\usepackage[math]{anttor}

%paper setup
\usepackage{geometry}
\geometry{letterpaper, portrait, margin=1in}
\usepackage{fancyhdr}

%symbols
\usepackage{amsmath}
\usepackage{amssymb}
\usepackage{hyperref}
\usepackage{gensymb}

\usepackage[T1]{fontenc}
\usepackage[utf8]{inputenc}

%chemistry stuff
\usepackage[version=4]{mhchem}
\usepackage{chemfig}

%plotting
\usepackage{pgfplots}
\usepackage{tikz}

%\usepackage{natbib}

%graphics stuff
\usepackage{graphicx}
\graphicspath{ {./images/} }

%a useful command
\newcommand{\plain}[1]{\textrm{#1}}

%code stuff
%when using minted, make sure to add the -shell-escape flag
%you can use lstlisting if you don't want to use minted
%\usepackage{minted}
%\usemintedstyle{pastie}
%\newminted[javacode]{java}{frame=lines,framesep=2mm,linenos=true,fontsize=\footnotesize,tabsize=3,autogobble,}
%\newminted[cppcode]{cpp}{frame=lines,framesep=2mm,linenos=true,fontsize=\footnotesize,tabsize=3,autogobble,}

\usepackage{listings}
\usepackage{color}
\definecolor{dkgreen}{rgb}{0,0.6,0}
\definecolor{gray}{rgb}{0.5,0.5,0.5}
\definecolor{mauve}{rgb}{0.58,0,0.82}

\lstset{frame=tb,
	language=Java,
	aboveskip=3mm,
	belowskip=3mm,
	showstringspaces=false,
	columns=flexible,
	basicstyle={\small\ttfamily},
	numbers=none,
	numberstyle=\tiny\color{gray},
	keywordstyle=\color{blue},
	commentstyle=\color{dkgreen},
	stringstyle=\color{mauve},
	breaklines=true,
	breakatwhitespace=true,
	tabsize=3
}
% text + color boxes
\usepackage{tcolorbox}
\newtcolorbox{mathbox}[1]{title = {#1}}

\pagestyle{fancy}
\fancyhf{}
\rhead{Avinash Iyer}
\lhead{Calculating GDP Worksheet}
\begin{document}{
Using the GDP table below, we can calculate various quantities:
  \begin{center}
    \begin{tabular}{c|c|c|c|c}
      Year & Price of Apples & Quantity of Apples & Price of Computers & Quantity of Computers\\
      \hline
      2026 & \$2 & 500 & \$1000 & 5 \\
      2027 & \$3 & 550 & \$ 1000 & 6
      \end{tabular}
  \end{center}
  \begin{itemize}
    \item \textbf{Nominal GDP}
        \begin{itemize}
          \item Nominal GDP in 2026:
            \[2\times 500 + 1000 \times 5 = \$6000\]
          \item Nominal GDP in 2027:
            \[3\times 550 + 1000 \times 6 = \$7650\]
        \end{itemize}
    \item \textbf{Real GDP, 2026 prices}
        \begin{itemize}
          \item Real GDP in 2026 using 2026 prices:
            \[2\times 500 + 1000\times 5 = \$6000\]
         \item Real GDP in 2027 using 2026 prices:
              \[2\times 550 + 1000\times 6 = \$7100\]
        \end{itemize}
    \item \textbf{Real GDP, 2027 prices}
          \begin{itemize}
          \item Real GDP in 2026 using 2027 prices:
            \[3\times 500 + 1000\times 5 = \$6500\]
          \item Real GDP in 2027 using 2027 prices:
              \[3\times 500 + 1000\times 6 = \$7650\]
          \end{itemize}
        \item \textbf{Change in Real GDP (Laspeyres Index)}
          \[\frac{7100-6000}{6000}\times 100 = 18.3\%\]
        \item \textbf{Change in Real GDP (Paasche Index)}
          \[\frac{7650-6500}{6500}\times 100 = 17.7\%\]
        \item \textbf{Fisher Index}
          \[\frac{18.3\% + 17.7\%}{2} = 18\%\]
        \item \textbf{Real GDP in 2026 in Chained Prices, benchmarked to 2027}
          \[\frac{7650}{1.18} = \$6483\]
        \item \textbf{Inflation Rate}
          \[\frac{7650-6483}{6483} = 18\%\]
  \end{itemize}
}\end{document}
