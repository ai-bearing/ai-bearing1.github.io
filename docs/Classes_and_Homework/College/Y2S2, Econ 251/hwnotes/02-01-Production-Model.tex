\documentclass[8pt]{extarticle}
\title{}
\author{Avinash Iyer}
\date{}

%font setup
%
%\usepackage[math]{anttor}

%paper setup
\usepackage{geometry}
\geometry{letterpaper, portrait, margin=1in}
\usepackage{fancyhdr}

%symbols
\usepackage{amsmath}
\usepackage{amssymb}
\usepackage{hyperref}
\usepackage{gensymb}

\usepackage[T1]{fontenc}
\usepackage[utf8]{inputenc}

%chemistry stuff
\usepackage[version=4]{mhchem}
\usepackage{chemfig}

%plotting
\usepackage{pgfplots}
\usepackage{tikz}

%\usepackage{natbib}

%graphics stuff
\usepackage{graphicx}
\graphicspath{ {./images/} }

%a useful command
\newcommand{\plain}[1]{\textrm{#1}}

%code stuff
%when using minted, make sure to add the -shell-escape flag
%you can use lstlisting if you don't want to use minted
%\usepackage{minted}
%\usemintedstyle{pastie}
%\newminted[javacode]{java}{frame=lines,framesep=2mm,linenos=true,fontsize=\footnotesize,tabsize=3,autogobble,}
%\newminted[cppcode]{cpp}{frame=lines,framesep=2mm,linenos=true,fontsize=\footnotesize,tabsize=3,autogobble,}

\usepackage{listings}
\usepackage{color}
\definecolor{dkgreen}{rgb}{0,0.6,0}
\definecolor{gray}{rgb}{0.5,0.5,0.5}
\definecolor{mauve}{rgb}{0.58,0,0.82}

\lstset{frame=tb,
	language=Java,
	aboveskip=3mm,
	belowskip=3mm,
	showstringspaces=false,
	columns=flexible,
	basicstyle={\small\ttfamily},
	numbers=none,
	numberstyle=\tiny\color{gray},
	keywordstyle=\color{blue},
	commentstyle=\color{dkgreen},
	stringstyle=\color{mauve},
	breaklines=true,
	breakatwhitespace=true,
	tabsize=3
}
% text + color boxes
\usepackage{tcolorbox}
\newtcolorbox{mathbox}[1]{title = {#1}}

\pagestyle{fancy}
\fancyhf{}
\rhead{Avinash Iyer}
\lhead{Solving the Production Model}
\begin{document}{
\section*{Equations}%
  \begin{itemize}
    \item $Y = \overline{A}K^{1/3}L^{2/3}$
    \item $r = \frac{1}{3}\frac{Y}{K}$
    \item $w = \frac{2}{3}\frac{Y}{L}$
    \item $K = \overline{K}$
    \item $L = \overline{L}$
  \end{itemize}
\section*{Unknowns}%
  \begin{itemize}
    \item $Y$
    \item $K$
    \item $L$
    \item $r$
    \item $w$
  \end{itemize}
We want to solve for these variables using only values we know, which are variables that have bars over them.
\section*{Solution of the Production Model}%
  \begin{itemize}
    \item $K^{*} = \overline{K}$
    \item $L^{*} = \overline{L}$
    \item $Y^{*} = \overline{A}\overline{K}^{1/3}\overline{L}^{2/3}$
    \item $r^{*} = \frac{1}{3}\frac{Y^{*}}{K^{*}} = \frac{1}{3}\overline{A}\left(\frac{\overline{L}}{\overline{K}}\right)^{2/3}$
    \item $w^{*} = \frac{2}{3}\frac{Y^{*}}{L^{*}} = \frac{2}{3}\overline{A}\left(\frac{\overline{K}}{\overline{L}}\right)^{1/3}$
  \end{itemize}
\section*{Insights}%
  \begin{itemize}
    \item We can gain a lot of insight from the production function about what makes a country rich or poor.
    \item We know that $Y$ is dependent on $\overline{A}$, or Total Factor Productivity, $\overline{K}$, or capital stock, and $\overline{L}$, or labor.
    \item We care primarily about output per worker as a measure of wealth --- this is found as $y^{*} := Y^{*}/L^{*} = \overline{A}\left(\frac{\overline{K}}{\overline{L}}\right)^{1/3} = \overline{A}k^{1/3}$ where $k:= \overline{K}/\overline{L}$, or capital per worker.
    \item We can then find that some countries are richer than others either when they have more capital per worker or they have a higher value of total factor productivity.
  \end{itemize}
}\end{document}
