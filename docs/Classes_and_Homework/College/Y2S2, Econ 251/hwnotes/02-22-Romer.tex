\documentclass[11pt]{extarticle}
\title{}
\author{Avinash Iyer}
\date{}
\usepackage[shortlabels]{enumitem}

%font setup
%
%\usepackage[math]{anttor}

%paper setup
\usepackage{geometry}
\geometry{letterpaper, portrait, margin=1in}
\usepackage{fancyhdr}

%symbols
\usepackage{amsmath}
\usepackage{mathtools}
\usepackage{amssymb}
\usepackage{hyperref}
\usepackage{gensymb}

\usepackage[T1]{fontenc}
\usepackage[utf8]{inputenc}

%chemistry stuff
\usepackage[version=4]{mhchem}
\usepackage{chemfig}

%plotting
\usepackage{pgfplots}
\usepackage{tikz}

%\usepackage{natbib}

%graphics stuff
\usepackage{graphicx}
\graphicspath{ {./images/} }

%code stuff
%when using minted, make sure to add the -shell-escape flag
%you can use lstlisting if you don't want to use minted
%\usepackage{minted}
%\usemintedstyle{pastie}
%\newminted[javacode]{java}{frame=lines,framesep=2mm,linenos=true,fontsize=\footnotesize,tabsize=3,autogobble,}
%\newminted[cppcode]{cpp}{frame=lines,framesep=2mm,linenos=true,fontsize=\footnotesize,tabsize=3,autogobble,}

\usepackage{listings}
\usepackage{color}
\definecolor{dkgreen}{rgb}{0,0.6,0}
\definecolor{gray}{rgb}{0.5,0.5,0.5}
\definecolor{mauve}{rgb}{0.58,0,0.82}

\lstset{frame=tb,
	language=Java,
	aboveskip=3mm,
	belowskip=3mm,
	showstringspaces=false,
	columns=flexible,
	basicstyle={\small\ttfamily},
	numbers=none,
	numberstyle=\tiny\color{gray},
	keywordstyle=\color{blue},
	commentstyle=\color{dkgreen},
	stringstyle=\color{mauve},
	breaklines=true,
	breakatwhitespace=true,
	tabsize=3
}
% text + color boxes
\usepackage{tcolorbox}
\newtcolorbox{problem}[1]{colback = white, title = {#1}}
\newtcolorbox{solution}{colback = white, colframe = black!75!white, title = Solution}
%including PDFs
\usepackage{pdfpages}
\setlength{\parindent}{0pt}

\pagestyle{fancy}
\fancyhf{}
\rhead{Avinash Iyer}
\lhead{Romer's Growth Model}
\begin{document}{
\begin{problem}{Solving Romer Model}
  \begin{align*}
    \shortintertext{Output per capita:}
    y_t &= \frac{Y_t}{\overline{L}} \\
    \shortintertext{Production Function:}
    y_t &= \frac{A_t L_{yt}}{\overline{L}} \\
    \shortintertext{Resource Constraint:}
    L_{at} + L_{yt} &= \overline{L} \\
    \shortintertext{Resource Allocation:}
    L_{at} &= \overline{\ell}\overline{L} \\
    L_{yt} &= (1-\overline{\ell})\overline{L}
    \shortintertext{Bringing to terms:}
    y_{t} &= \frac{A_t(1-\overline{\ell})\overline{L}}{\overline{L}} \\
          &= A_t (1-\overline{\ell})
  \end{align*}
This is nice so far, but our primary problem is that $A_t$ is an endogenous variable, and we would ideally like to see our output per capita as a function of exogenous variables only.
  \begin{align*}
    \shortintertext{Change in $\overline{A}$:}
    \Delta A_t &= \overline{z} A_t L_{at} \\
    \frac{\Delta A_t}{A_t} &= \frac{\overline{z} A_t L_{at}}{A_t} \\
                           &= \overline{z} L_{at} \\
                           &= \overline{z} \overline{\ell} \overline{L}\\ 
    g_a &= \overline{z} \overline{\ell} \overline{L}
  \end{align*}
  Therefore, because we now know that $g_a = \overline{z}\overline{\ell}\overline{L}$, and that $A_t = \overline{A_0}(1 + g_a)^t$, we can solve for output per capita:
  \[
    \boxed{y_t = \overline{A_0}(1-\overline{\ell})(1 + \overline{z}\overline{\ell}\overline{L})^t}
  \]
  We can see that output increases consistently, as opposed to the Solow model (which reaches a steady state). We need to plot such a function on a ratio scale, and we find that we have long run growth so far as there are new ideas and technologies.\\

  In the Solow model, we run into diminishing returns --- additions to stock of capital reduce in marginal effectiveness, up until all the new capital being added to the economy is simply to replace depreciated capital. However, with Romer's model, ideas do not face marginal diminishing returns.
\end{problem}
}\end{document}
